\documentclass[10pt]{article}
\renewcommand{\baselinestretch}{1}

\usepackage{lscape,verbatim}
\usepackage{graphics,amsmath,pstricks}
\usepackage{amssymb,enumerate}
\usepackage{amsbsy,amsmath,amsthm,amsfonts, amssymb}
\usepackage{graphicx, rotate, array}
\usepackage{geometry,multirow}
\usepackage{float}
%\usepackage{hyperref}
\usepackage{natbib}
\newcommand{\HRule}{\rule{\linewidth}{0.5mm}}
\usepackage{mathrsfs}
\usepackage{algpseudocode} % algs
\usepackage{algorithm} % algs
\algrenewcommand\algorithmicindent{0.4em}
\usepackage[english]{isodate} % nice date
\cleanlookdateon
\newtheorem{lemma}{Lemma}
\newtheorem{theorem}{Theorem}


%\renewcommand{\familydefault}{cmss}
\textwidth=6.65in \textheight=9.5in
\parskip=.025in
\parindent=0.1in
\oddsidemargin=-0.1in \evensidemargin=-.1in \headheight=-.5in
\footskip=0.6in \DeclareMathOperator*{\argmax}{argmax}
\DeclareMathOperator*{\argmin}{argmin}
\DeclareMathOperator*{\logit}{logit}

\author{Brian Williamson}
\title{Simulation plan: clinical SLAPNAP manuscript}
\date{\today}

\begin{document}
\maketitle

\section{Introduction}

This document describes the planned simulations for the clinical SLAPNAP manuscript. The simulations address the following goals:
\begin{enumerate}
    \item determine if using external data (i.e., Env sequences) results in variance reduction for estimating R-squared and AUC; and
    \item determine if using SLAPNAP can improve sieve analysis.
\end{enumerate}
In the next section, I describe my proposed approach to answering these two questions using simulations.

\section{Simulation plan}
\subsection{Using auxiliary response variables reduces estimation variance}
\subsubsection{Continuous outcome}
The simplest case for examining variance reductions is in the case of a continuous outcome (quantitative IC$_{80}$). The results from SLAPNAP runs predicting IC$_{80}$ for each bnAb in Table 1 of the SLAPNAP clinical manuscript are presented in the second column of Table~\ref{tab:perf}.

For this simulation, for each bnAb in Table~\ref{tab:perf}, we denote by $X$ the $\log_{10}$ PAR score obtained from SLAPNAP and by $Y$ the $\log_{10}$ IC$_{80}$ readout. We suppose that $(X,Y) \sim N(\mu, \Sigma)$, where $\mu = (\mu_1, \mu_2)$ is a mean vector and $\Sigma = \begin{bmatrix} \Sigma_{11} & \Sigma_{12} \\ \Sigma_{21} & \Sigma_{22} \end{bmatrix}$ is a covariance matrix. For simplicity, we set $\mu_1 = \mu_2 = \mu^*$, with $\mu^*$ the mean IC$_{80}$ value in the CATNAP data for the specified bnAb (or bnAb regimen), and set $\Sigma_{11} = \Sigma_{22} = \Sigma^*$, with $\Sigma^*$ the variance of IC$_{80}$ in the CATNAP data for the specified bnAb (or bnAb regimen). Under this model, we can write
\begin{align*}
    R^2 =& \ 1 - \frac{var(X)}{var(Y)} \\
    =& \ 1 - \frac{\Sigma_{12}^2\Sigma_{22}^{-1}}{\Sigma_{11}}.
\end{align*}
We can then generate data according to $\Sigma_{12}^2 = (1 - R^2)\Sigma_{22}\Sigma_{11}$, using the point estimates of CV-$R^2$ provided in Table~\ref{tab:perf} (and under the simplifying assumption made above, $\Sigma_{11} = \Sigma_{22} = \Sigma^*$).

For each bnAb, we generate 1000 random datasets according to the following specification. We generate $n = 1000$ IC$_{80}$ values and corresponding Env sequences from a $N(\mu, \Sigma)$ distribution, resulting in $X$ and $Y$. We truncate $Y$ at 1 (equivalent to an IC$_{80}$ of 10), since high values of IC$_{80}$ are difficult to quantify. We then generate $(1 + \epsilon)n$ additional Env sequences from this same distribution without IC$_{80}$ measured, where $\epsilon \in \{0.5, 1, 2\}$; we call these data $W$. The final dataset is then $(W, X, Y)$. We will generate 3000 total replications, 1000 for each value of $\epsilon$.

For each random dataset, we estimate the mean of $Y$ using $(X,Y)$ and using $(W,X,Y)$. In other words, setting $U = (X,Y)$ (yielding $2n$ observations of $U$) and $V = (W,X,Y)$ (yielding $n(3 + \epsilon)$ observations of $V$), we compute
\begin{align*}
    \overline{Y}_1 =& \ \frac{1}{2n}\sum_{i=1}^{2n}U_i \\
    \overline{Y}_2 =& \ \frac{1}{n(3 + \epsilon)}\sum_{i=1}^{n(3 + \epsilon)}V_i;\\
\end{align*}
since both $X_i$ and $W_i$ are predictions of $Y_i$, each provides us with information about the outcome. We then compute the Monte-Carlo variance of the estimated means $\overline{Y}_1$ and $\overline{Y}_2$ over the 1000 replications, and estimate the relative efficiency of using $W$ by taking the ratio of the Monte-Carlo variance ignoring $W$ to the Monte-Carlo variance using $W$.

\begin{table}
    \centering
    \caption{Prediction performance for each broadly neutralizing antibody (bnAb) or bnAb regimen from Table 1 of the SLAPNAP clinical manuscript. For each SLAPNAP run, performance is measured using cross-validated R-squared (CV-$R^2$) for continuous IC$_{80}$ and using cross-validated AUC (CV-AUC) for (estimated) sensitivity and multiple sensitivity (where applicable).}
    \begin{tabular}{l|ccc}
        & \multicolumn{3}{c}{Prediction performance} \\
        bnAb (or bnAb regimen) & IC$_{80}$ & (Estimated) Sensitivity & Multiple Sensitivity \\
        \hline
        VRC01 & 0.345 & 0.744 & -- \\
        PGT121 & 0.571 & 0.85 & -- \\
        VRC07-523-LS & 0.193 & 0.728 & --\\
        VRC07-523-LS + 10-1074 & 0.319 & 0.783 & 0.784 \\
        VRC07-523-LS + PGT121 & 0.316 & 0.768 & 0.781 \\
        VRC07-523-LS + PGDM1400 & 0.255 & 0.638 & 0.669 \\
        VRC07-523-LS + PGT121 + PGDM1400 & 0.181 & 0.73 & 0.708 \\
        VRC01/PGDM1400/10e8v4 & 0.254 & 0.81 & --
    \end{tabular}
    \label{tab:perf}
\end{table}

\subsubsection{Binary outcome}

We use a similar approach for binary outcomes to the approach outlined in the previous section. We now denote by $Y$ the indicator that IC$_{80} < 1$, and denote by $X$ the $\log_{10}$ PAR score. We suppose that $Y \sim Bern(p)$, where $p$ is the sample proportion for each bnAb in the CATNAP data, and that $X \mid Y = y \sim N(\mu_y, \sigma^2_y)$. Based on this specification, we can write (for two iid samples $(X_1, Y_1)$ and $(X_2, Y_2)$)
\begin{align*}
    AUC =& P(X_1 < X_2 \mid Y_1 = 0, Y_2 = 1) \\
    =& \ P(X_2 - X_1 > 0 \mid Y_1 = 0, Y_2 = 1);
\end{align*}
setting $Z = X_2 - X_1$, we see that $Z$ has a normal distribution with mean $\mu_1 - \mu_0$ and variance $\sigma^2_0 + \sigma^2_1$. Thus,
\begin{align*}
    AUC =& \ P\left(\frac{Z - (\mu_1 - \mu_0)}{\sqrt{\sigma^2_0 + \sigma^2_1}} > \frac{- (\mu_1 - \mu_0)}{\sqrt{\sigma^2_0 + \sigma^2_1}}\right) \\
    \Rightarrow \frac{(\mu_1 - \mu_0)}{\sqrt{\sigma^2_1 + \sigma^2_0}} =& \ (-1)\Phi^{-1}(1 - AUC),
\end{align*}
where $\Phi$ denotes the standard normal cdf. Thus, setting $\sigma^2_1 = \sigma^2_0 = 0.005$ and $\mu_0 = -0.32$ (corresponding to a mean PAR score of 0.47 among those with $Y = 0$), we can generate $n$ iid copies of $X$ and $Y$, and generate $n(1 + \epsilon)$ iid copies of $W$ (the additional Env sequences).

For each random dataset, we again create vectors $U$ and $V$, where now $U = (I(10 ^ X > 0.5), Y)$ and $V = (I(10 ^ X > 0.5), I(10^W > 0.5), Y)$. Then we compute
\begin{align*}
    \overline{P}_1 =& \ \frac{1}{2n}\sum_{i=1}^{2n} U_i \text{ and } \\
    \overline{P}_2 =& \ \frac{1}{n(3+\epsilon)}\sum_{i=1}^{n(3+\epsilon)}V_i;
\end{align*}
these estimate the probability that IC$_{80} < 1$. We then compute the Monte-Carlo variance of the estimated means over the 1000 replications, and estimate the relative efficiency of using $W$ by taking the ratio of the Monte-Carlo variance ignoring $W$ to the Monte-Carlo variance using $W$.

\subsection{Sieve analysis}

The goal of this simulation is to see if SLAPNAP improves sieve analysis; in particular, we want to see if using SLAPNAP results in improved power for detecting sieve effects. For simplicity, we will focus on VRC01 only for this simulation.

We will use Cox modeling (Lunn and McNeil, 1995) to estimate differential prevention efficacy by `putative resistant' genotype versus `other genotype' for each of the 26 positions in gp120 identified as important for the AMP sieve analysis with enough variability in the AMP data, defined by at least 4 primary endpoint cases having a sequence with a minority residu at the given position (harmonized with the AMP sieve analysis). We denote the sites that pass the variability threshold by $\{H_i\}_{i=1}^I$, where $I \leq 26$.

Next, we define the alternative hypotheses for sieve effects at each site. We assume an overall prevention efficacy of 18\% (as identified in AMP), i.e., set $PE(\text{overall}) = 0.18$. We also assume $PE(\text{putative resistant genotype at site } H_i) = 0$ for all $i$. Then for each site $H_1,\ldots,H_I$, we set
\begin{align*}
    PE(\text{overall}) = \gamma_i PE(\text{putative resistant genotype at site } H_i) + (1 - \gamma_i)PE(\text{other genotype at site } H_i),
\end{align*}
where $\gamma_i$ is the fraction of placebo-arm AMP viruses that have the putative resistant genotype. This allows us to determine the alternative hypothesis for each site. For all other gp120 sites, we assume the null hypothesis $PE(\text{putative resistant genotype at site } H_i) = PE(\text{other genotype at site } H_i)$. We compute each $\gamma$ using the AMP data.

For each of 1000 Monte-Carlo replications, we generate data for each site in gp120 and apply a standard Lunn and McNeil test for differential PE at each gp120 site with two-sided level 0.05; we define a detection as the resulting p-value from this test being less than the multiplicity-adjusted threshold of 0.05 / (number of sites in gp120 with enough variability). We compute the power as the Monte-Carlo average number of true detections.

We also perform a separate 1000 Monte-Carlo replications, generating data for each site in gp120 and apply a standard Lunn and McNeil test for differential PE at each of the important high-variability sites with two-sided level 0.05; we define a detection as the resulting p-value from this test being less than the multiplicity-adjusted threshold of 0.05 / 26. We compute the power as the Monte-Carlo average number of true detections.

Comparing the power of these two approaches (analyze all of gp120, analyze only the important high-variability sites) will provide a useful description of how much SLAPNAP aids in sieve analysis.
\end{document}
